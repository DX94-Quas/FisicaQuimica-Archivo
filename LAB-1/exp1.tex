\documentclass[11pt, letterpaper]{article}
\usepackage[utf8]{inputenc}
\usepackage{fontspec}
\usepackage{geometry}
\usepackage[backend=biber]{biblatex}
\addbibresource{Lab1.bib}
\title{Laboratorio 1 - Ley de los Gases}
%\date{}
 
\begin{document}
\maketitle
\section{Introducción}
\subsection{Gas Ideal}
\subsection{Ley de Boyle-Mariotte}
vbudbfdugbfdyugbdgfybdgubfygbdu
\subsection{Ley de Charles}
\subsection{Ley de Gay-Lussac}
\section{Objetivos}
\subsection{Generales}
asdffff\cite{IFixit2016}
asdhsfsufvudgybdgufdbgydbgfdugbfd
fgudfigdbigfdbfgdf
\subsection{Específicos}
\section{Materiales y Métodos}
\subsection{Materiales}
\subsubsection{Experimento de Boyle-Mariotte}
\subsubsection{Experimento de Charles}
\subsubsection{Experimento de Gay-Lussac}
\subsection{Métodos}
\subsubsection{Experimento de Boyle-Mariotte}
\subsubsection{Experimento de Charles}
\subsubsection{Experimento de Gay-Lussac}
\section{Resultados}
\subsection{Experimento de Boyle-Mariotte}
\begin{table}[h!]
\centering
\begin{tabular}{||c c c c||} 
 \hline
 Muestra & Col2 & Col2 & Col3 \\ [0.5ex] 
 \hline\hline
 1 & 6 & 87837 & 787 \\ 
 2 & 7 & 78 & 5415 \\
 3 & 545 & 778 & 7507 \\
 4 & 545 & 18744 & 7560 \\
 5 & 88 & 788 & 6344 \\ [1ex] 
 \hline
\end{tabular}
\caption{Table to test captions and labels}
\label{table:1}
\end{table}
\subsection{Experimento de Charles}
abcdefghijklmnopqrstuvwxyz
\subsection{Experimento de Gay-Lussac}
\section{Conclusión}

\section{Bibliografía}
\printbibliography 
\end{document}