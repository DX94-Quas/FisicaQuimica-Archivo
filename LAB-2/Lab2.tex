\documentclass[11pt, letterpaper]{article}
\usepackage[utf8]{inputenc}
\usepackage{fontspec}
\usepackage{array}
\usepackage{geometry}
\usepackage{booktabs}
\usepackage{colortbl}
\usepackage{graphicx}
\graphicspath{ {.} }

\newenvironment{conditions}
  {\par\vspace{\abovedisplayskip}\noindent\begin{tabular}{>{$}l<{$} @{${}={}$} l}}
  {\end{tabular}\par\vspace{\belowdisplayskip}}
%\usepackage[backend=biber]{biblatex}
%\addbibresource{Lab1.bib}
\title{Laboratorio 2 - Líquidos}
%\date{}

\author{
  Rodrigo Sandoval\\
  \and
  Gerardo Viscarra\\
  \and
  Osman Antelo\\
  \and
  David Aguilera\\
  \and
  Alfredo Gutierrez\\
}
 
\begin{document}
\maketitle
\section{Introducción}
Los líquidos desde un punto de vista teórico son considerados como   una continuación de la fase gaseosa, con ciertas propiedades que lo caracterizan de los demás estados, en las siguientes prácticas demostraremos dichas propiedades de los líquidos.\\

Para la determinación de viscosidad se utilizan los viscosímetros de Ostwald y Stokes.Para la determinación de la densidad, se puede utilizar el densímetro o tan bien se la puede calcular por picnometría.\\

Para la determinación de la tensión superficial existen varias prácticas sencillas
\section{Marco Teorico}
Los líquidos se caracterizan por que poseen volumen propio, se adaptan a la forma del recipiente que los contiene, pueden fluir libremente sin que tenga que aplicarse para ello una gran fuerza, ser muy poco compresibles y pasar al estado de vapor a cualquier temperatura.\\

Entre las propiedades de los líquidos tenemos:  viscosidad, tensión superficial, presión de vapor, etc.
\subsection{Viscosidad}
Es una medida de la resistencia interna al desplazamiento relativo de las capas sucesivas, debido a las fuerzas de atracción entre las  moléculas de esas capas.\\

Las dos principales formas para determinar la viscosidad son,   utilizando el viscosímetro de Ostwald, y el viscosímetro de Stokes.
\subsection{Viscosímetro de Ostwald}
se basa en el principio de flujo de fluidos y es utilizado para  medir  la  viscosidad  de  líquidos  no  muy  viscosos.  Consiste  en  comparar  el tiempo  de  fluido  de  un  líquido  (cuya  densidad  y  viscosidad  son  conocidas),  con  el tiempo de flujo de otro líquido cuya densidad  es conocida, la viscosidad se calcula con la siguiente ecuación:

\begin{equation}
\frac{\mu _{1}}{\mu_{2}} = \frac{t_{1} \rho_{1} }{t_{2} \rho_{2}}
\end{equation}

Donde:
\begin{conditions}
 \mu _{1} &  Viscosidad de Liquido 1 \\
 \mu _{2} &  Viscosidad de Liquido 2 \\   
 t_{1} &  Tiempo de caída de Liquido 1 \\
 \rho_{1} &  Densidad de Liquido 1 \\
 t_{2} &  Tiempo de caída de Liquido 2 \\
 \rho_{2} &  Densidad de Liquido 2
\end{conditions}

\subsection{Viscosímetro de Stokes}
se fundamenta en la caída libre de esferas a través de un líquido  viscoso en reposo, básicamente consiste en un tubo de vidrio lleno  del líquido que se quiere determinar su viscosidad, se deja caer la esfera en una altura determinada y con el tiempo que tarda en  recorrer la esfera, dicha altura y conociendo la densidad tanto del líquido como del sólido se calcula la viscosidad con la siguiente ecuación

\begin{equation}
V_{s} = \frac{2}{9} \frac{r^{2} g (\rho_{p} -\rho_{f})}{\eta}
\end{equation}

Donde:
\begin{conditions}
 V_{s} &  Velocidad de Caída \\
 g &  Gravedad \\   
 \rho_{p} &  Densidad de la partícula \\
 \eta &  Viscosidad del fluido \\
 r &  Radio de la Partícula \\
 \rho_{f} &  Densidad del fluido
\end{conditions}

\subsection{Tension superficial}
Es la tendencia de un líquido a reducir a lo mínimo el área de su superficie por atracción molecular.\\

La  tensión  molecular  se  define  como  la  fuerza  que  actúa  sobre  la  línea  de  un centímetro de longitud sobre la superficie. La tendencia de líquido a contraerse se debe a que las moléculas que se encuentran dentro del líquido atraen a las moléculas de la superficie.\\

Cuando un líquido descansa sobre una superficie sólida, el ángulo entre un líquido y el  sólido  se  llama  ángulo  de  contacto,  cuando  este  ángulo  es  menor  a  90º,  se  dice que el líquido moja al sólido. Cuando el ángulo aumenta, las fases de atracción entre las fases distintas disminuyen. La tensión superficial se puede calcular con la siguiente ecuación:

\section{Experiencia - Determinacion de la viscosidad mediante  viscosímetro de Stokes}
\subsection{Objetivos}
\subsubsection{Generales}
El objetivo de esta práctica es determinar la viscosidad de un líquido
utilizando un viscosímetro.
\subsubsection{Específicos}
\begin{itemize}
\item Determinar la viscosidad por medio del viscosímetro de Stokes.
\item Comparar el valor experimental con el valor teórico, determinando el error relativo.
\end{itemize}
\subsection{Materiales}
\begin{itemize}
	\item Esferas metalicas
	\item Calibrador
	\item Tubo de ensayo
	\item Cronometro
\end{itemize}
\subsubsection{Sustancia a ser analizadas}
\begin{itemize}
	\item Agua
	\item Agua destilada
	\item Alcohol
	\item Glicerina
	\item Aceite
\end{itemize}
\subsubsection{Procedimiento}
\begin{itemize}
	\item anotar el diámetro de cada esfera usando el calibrador y registrar
	\item pesar cada esfera y registrar
	\item llenar el viscosimetro de liquido problema
	\item con el cronometro anotar cuanto
\end{itemize}
\subsection{Calculos}
\begin{itemize}
	\item alto de la probeta - 230mm
	\item Diámetro de esfera grande - 1.545cm
	\item Diámetro de esfera mediana - 0.66cm
	\item Diámetro de esfera pequeña - 0.42cm
\end{itemize}
% Table generated by Excel2LaTeX from sheet 'Sheet3'
\begin{center}
    \begin{tabular}{|p{6.93em}|r|r|r|r|}
    \toprule
    \multicolumn{5}{|p{38.285em}|}{Esfera Pequeña} \\
    \midrule
    liquido & \multicolumn{1}{p{6.57em}|}{tiempo(seg)} & \multicolumn{1}{p{6.57em}|}{densidad liq(kg/m3)} & \multicolumn{1}{p{9.5em}|}{visc exp(kg/ms)} & \multicolumn{1}{p{8.715em}|}{visc teorica(kg/ms)} \\
    \midrule
    agua  & 0.54  & 1000  & 739.357 & 0.001050 \\
    \midrule
    alcohol & 0.43  & 825.4 & 603.662 & 1.392300 \\
    \midrule
    glicerina & 3.08  & 1.26  & 4040.848 & 0.391000 \\
    \midrule
    aceite & 0.63  & 880   & 877.972 & 0.001783 \\
    \bottomrule
    \end{tabular}%
  \label{tab:addlabel}%
\end{center}%
% Table generated by Excel2LaTeX from sheet 'Sheet3'
\begin{center}
    \begin{tabular}{|l|r|r|}
\cmidrule{2-3}    \multicolumn{1}{c|}{} & \multicolumn{1}{c|}{Densidad (kg/m3)} & \multicolumn{1}{c|}{Viscosidad (kg/m·s)} \\
    \midrule
    agua  & 1000  & 0.00105 \\
    \midrule
    glicerina & 1.26  & 1.3923 \\
    \midrule
    aceite & 880   & 0.391 \\
    \midrule
    alcohol & 825.4 & 0.001783 \\
    \bottomrule
    \end{tabular}%
  \label{tab:addlabel}%
\end{center}%
% Table generated by Excel2LaTeX from sheet 'Sheet3'
\begin{center}
    \begin{tabular}{|r|l|}
    \toprule
    0.23  & distancia(mt) \\
    \midrule
    7880  & densidad de bola(kg/m3) \\
    \bottomrule
    \end{tabular}%
  \label{tab:addlabel}%
\end{center}%
% Table generated by Excel2LaTeX from sheet 'Sheet3'
\begin{center}
    \begin{tabular}{|l|r|r|r|r|}
\cmidrule{2-5}    \multicolumn{1}{r|}{} & \multicolumn{4}{c|}{Tiempos(seg)} \\
    \midrule
    \multicolumn{1}{|c|}{Material} & \multicolumn{1}{c|}{Alcohol} & \multicolumn{1}{c|}{Glicerina} & \multicolumn{1}{c|}{Aceite} & \multicolumn{1}{c|}{Agua} \\
    \midrule
    pequeña & 0.43  & 3.08  & 0.63  & 0.54 \\
    \midrule
    mediana & 0.35  & 2.01  & 0.53  & 0.36 \\
    \midrule
    grande &       & 1.74  &       &  \\
    \midrule
    \multicolumn{1}{r|}{} & \multicolumn{4}{c|}{Velocidades(m/seg)} \\
    \midrule
    pequeña & 0.535 & 0.075 & 0.365 & 0.426 \\
    \midrule
    mediana & 0.657 & 0.114 & 0.434 & 0.639 \\
    \midrule
    grande &       & 0.132 &       &  \\
    \bottomrule
    \end{tabular}%
  \label{tab:addlabel}%
\end{center}%
% Table generated by Excel2LaTeX from sheet 'Sheet3'
\begin{center}
    \begin{tabular}{|l|r|r|}
\cmidrule{2-3}    \multicolumn{1}{r|}{} & \multicolumn{1}{l|}{diam(mt)} & \multicolumn{1}{l|}{rad(mt)} \\
    \midrule
    grande & 0.001545 & 0.0007725 \\
    \midrule
    mediana & 0.00066 & 0.00033 \\
    \midrule
    pequena & 0.00042 & 0.00021 \\
    \bottomrule
    \end{tabular}%
  \label{tab:addlabel}%
\end{center}%



\section{Experiencia - Tensión Superficial}
\subsection{Objetivos}
\subsubsection{Generales}
Descubrir y comprender las propiedades de la tensión superficial de los líquidos, demostrando con prácticas sencillas las manifestaciones de la tensión superficial.
\subsubsection{Específicos}
\begin{itemize}
	\item Observar la propiedad de los líquidos, donde su superficie se comporta como una membrana invisible.
	\item Determinar la tensión superficial por el método del tubo capilar.
	\item Determinar las variables de las que depende la tensión superficial de los líquidos.
\end{itemize}
\subsection{Materiales}
\begin{itemize}
	\item Cristalizador
	\item Moneda
	\item Dinamómetro
	\item Aguja
	\item Anillo de Aluminio
\end{itemize}
\subsubsection{Sustancias a ser analizadas}
\begin{itemize}
	\item Agua
	\item Agua destilada
	\item Alcohol
	\item Glicerina
	\item Aceite
\end{itemize}
\subsection{Procedimiento}
\begin{itemize}
	\item Colocar el dinamómetro al anillo de aluminio
	\item Llenar el cristalizador con el liquido
	\item Colocar el anillo sobre la superficie del liquido, y luego sacarlo
	\item Volver a colocar el anillo y sacarlo, esta vez intentar romper la película con la aguja
	\item anotar la fuerza que ejerce la película sobre el anillo, el comportamiento de la película y si la película fue rota por la punta de la aguja
	\item intentar poner la moneda sobre el liquido y anotar si esta flota o a que velocidad se hunde
\end{itemize}
\subsection{Calculos y Datos}
\begin{center}
% Table generated by Excel2LaTeX from sheet 'Sheet1'
    \begin{tabular}{|p{7.5em}|p{7.5em}|p{7.5em}|p{7.5em}|p{7.5em}|}
    \toprule
    \rowcolor[rgb]{ .776,  .878,  .706} \multicolumn{1}{|c|}{Agua} & \multicolumn{1}{|c|}{Alcohol} & \multicolumn{1}{|c|}{Agua destilada} & \multicolumn{1}{|c|}{Glicerina} & \multicolumn{1}{|c|}{Aceite} \\
    \midrule
    \rowcolor[rgb]{ 1,  1,  0} \multicolumn{5}{|c|}{Hoja circular} \\
    \midrule
    Se logra romper la pelicula con un portaminas, El dinamometro lee 0.05N. Se rompe la pelicula con el alfiler & Forma menos pelicula que el agua y el agua destilada, Dinamometro lee 0.049N, se rompe la pelicula antes de romperla con el alfiler & Al salir forma pelicula muy pequeña, el dinamometro lee 0.052N. No se puede romper la pelicula con el alfiler & Forma la pelicula mas larga de todas las sustancias, Dinamometro lee 0.055N. Aguja no rompe la pelicula & Forma poca pelicula, Se forma una pelicula adicional dentro del circulo, Diamometro lee 0.05N, Aguja no rompe la pelicula \\
    \midrule
    \rowcolor[rgb]{ 1,  1,  0} \multicolumn{5}{|c|}{Moneda} \\
    \midrule
    No flota, se hunde apenas colocarla  & No flota, se hunde apenas colocarla  & No flota, se hunde apenas colocarla  & Se hunde lentamente & No flota, se hunde apenas colocarla  \\
    \bottomrule
    \end{tabular}%
\end{center}

\section{Experiencia - Capilaridad}
Este experimento comparte los mismos objetivos de la experiencia anterior
\subsection{Materiales}
\begin{itemize}
	\item Tubos capilares
	\item Cristalizador
	\item Calibrador
\end{itemize}
\subsubsection{Sustancias a ser analizadas}
\begin{itemize}
	\item Agua
\end{itemize}
\subsubsection{Procedimiento}
\begin{itemize}
	\item Anotar el diámetro de cada tubo capilar
	\item Sumergir cada tubo capilar en el liquido 
	\item Sacar el tubo capilar del liquido y anotar a que altura quedo relleno con liquido una vez fuera del cristalizador
\end{itemize}
\subsection{Calculos y datos}
% Table generated by Excel2LaTeX from sheet 'Sheet2'
  \begin{center}
    \begin{tabular}{|c|c|}
    \toprule
    \rowcolor[rgb]{ .776,  .878,  .706} Altura del liquido(mm) & Diametro del tubo(mm) \\
    \midrule
    \rowcolor[rgb]{ 1,  1,  0} \multicolumn{2}{|c|}{Agua} \\
    \midrule
    31    & 1 \\
    \midrule
    16    & 2.5 \\
    \midrule
    6     & 5 \\
    \midrule
    1     & 8 \\
    \bottomrule
    \end{tabular}%
  \end{center}
\section{Conclusiones}
\begin{itemize}
	\item Todas las experiencias no se pudieron hacer o se cambiaron debido a la falta de material en el laboratorio
	\item No se logro verificar la viscosidad cuantitativa de los liquidos sino que se entendio a traves de la velocidad a la cual cae la esfera, ademas faltaron datos como la masa de las esferas, que se tuvo que asumir el material del que estaban hechas para obtener su densidad
	\item La capilaridad se logro demostrar pero por falta de tiempo no se intento en diferentes liquidos sino solo en el agua
	\item La tension superficial se logro observar satisfactoriamente, pero solo se observo con el anillo y una moneda, faltaron mas cuerpos y el dianmometro usado no tenia la resolucion suficiente
\end{itemize}
\section{Cuestionario}


\begin{enumerate}
   \item Indique que otros métodos existen para determinar la tensión superficial, explique cada uno de ellos.
   \begin{itemize}
     \item VISCOSÍMETRO ROTACIONAL ANALÓGICO - Instrumento de estructura compacta, de gran estabilidad en las medidas y alta exactitud y precisión, adecuado para lectura de viscosidades medias.
     \item VISCÓSIMETRO HOPPLER. - está basado en una modificación del Viscosímetro de bola, en donde una esfera rueda en el interior de un tubo que puede inclinarse un ángulo determinado. 
   \end{itemize}
   \item De que variables depende fundamentalmente la tensión superficial de los líquidos.
   \begin{itemize}
     \item La tensión superficial depende de la naturaleza del líquido, del medio que le rodea y de la temperatura
   \end{itemize}
   \item Qué efecto tiene en la tensión superficial un agente tenso activo.
   \begin{itemize}
     \item Son tambien llamados surfactante, sirven para reducir la tension superficial
   \end{itemize}
   \item Que influencia tiene la temperatura en la tensión superficial.
   \begin{itemize}
     \item En general, la tensión superficial disminuye con la temperatura, ya que las fuerzas de cohesión disminuyen al aumentar la agitación térmica
   \end{itemize}
   \item Investigue tres agentes tenso activos.
   \begin{itemize}
     \item Sulfatos y sulfonatos
    \item Fosfatos y fosfanatos
\item     Carboxilatos
   \end{itemize}
   \item Explique porque flota el corcho, porque se mantiene el corcho cerca de la pared del vaso cuando este no está completamente lleno.
   \begin{itemize}
     \item Debido a la adhesión entre las moléculas de agua y el vidrio, el nivel del agua es más alta en los bordes (el nivel de agua es cóncava). Como resultado, el corcho se mueve a los lados.
   \end{itemize}
   \item Explique porque los líquidos forman en las superficies meniscos cóncavos y convexos.
   \begin{itemize}
     \item Esto varia segun el grado de adhesion que tenga el liquido
   \end{itemize}
   \item Explique porque trepa el líquido por la cinta absorbente.
   \begin{itemize}
     \item por que la cinta esta hecho de fibras que se comportan como tubos capilares, por eso asciende el liquido por la cinta
   \end{itemize}
   \item Explique la diferencia que existe en los gráficos, altura vs. Tiempo del comportamiento del agua y la solución azucarada.
   \begin{itemize}
     \item la viscosidad afecta la velocidad a la que el liquido sube por la cinta absorbente
   \end{itemize}
   \item Explique la influencia de la viscosidad en el ascenso capilar.
   \begin{itemize}
     \item la viscosidad afecta a la velocidad a la que el liquido asciende, si el liquido es muy viscoso, le tomara mas tiempo subir por los capilares
   \end{itemize}
   
\end{enumerate}
\section{Fotos de experiencia}
\includegraphics[width=\textwidth]{1}\\

\includegraphics[width=\textwidth]{3}
\includegraphics[width=\textwidth]{4}
\includegraphics[width=\textwidth]{5}
\includegraphics[width=\textwidth]{6}
\includegraphics[width=\textwidth]{2}
\section{Bibliografía}
\begin{itemize}
\item https://www.i-ciencias.com/pregunta/4251/por-que-un-corcho-flotante-al-lado-de-un-vaso
\item https://en.wikipedia.org/wiki/Surfactant
\item http://www.sc.ehu.es/sbweb/fisica/fluidos/tension/introduccion/introduccion.htm
\item http://www.sc.ehu.es/sbweb/fisica3/fluidos/stokes/stokes.html
\end{itemize}
\end{document}